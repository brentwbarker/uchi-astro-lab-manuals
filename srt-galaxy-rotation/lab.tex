\chapter{How fast is the galaxy rotating and what does it mean?}

\section{Introduction}

\section{Learning Goals}
\begin{itemize}
	\item Use observation of hydrogen cloud velocities to determine motion of galactic plane
	
	\item Organize and present data in a logical and consistent manner and draw conclusions from it
	
	\item Interpret rotation curve data in order to make inferences about the mass distribution in the Milky Way
\end{itemize}

\section{Rotation of the Milky Way}

Throughout the past few weeks, you have been preforming labs which at times might have seemed disparate or only somewhat connected to each other. Now, you will bring all these concepts you have been working with in order to perform observations and make conclusions regarding the motion of the Milky Way. To do this, you will be measuring the line of sight velocity of Hydrogen clouds as they orbit the galactic plane and using this to calculate their circular velocities. We can take advantage of the geometry of the galaxy as shown in the figure below to determine the velocities of the hydrogen clouds.

To do this calculation, we will need several components. First, we will need the velocity of the sun in the line of sight for the galactic longitude $\gamma$. This is because the Sun is also moving along the galactic plane, and thus we need to be able to account for it in our measurements. We already know that the radial velocity of the sun is 220km, so using some simple trigonometry, we can see that the line of sight velocity is given by $$V_{sun}(\gamma) = 220\sin(\gamma)$$ We also need to account for Earths rotation around the sun as well as relative motion of the solar system. This is given to you by the SRT as the VLSR or Velocity relative to the Local Standard of Rest. From the graph generated by the telescope, we simply need the maximum VLSR, as it corresponds to the hydrogen cloud directly in our line of sight. The circular velocity of the cloud can thus be obtained by  $$V_c(r) = V_{max}(\gamma) + V_sun(\gamma)$$ Finally, for the final data processing, you will need the distance of that cloud from the center of the galaxy. Once again, from the geometry of the graph above, we can see that this can be found from the distance of the Sun to the center $r_0$ and the galactic longitude $\gamma$. The distance is then given by $$r = r_0 \sin(\gamma)$$



\subsection{Goal}

\subsection{Equipment}
\begin{itemize}
	\item LAB sky survey: \url{https://www.astro.uni-bonn.de/hisurvey/AllSky_profiles/}
\end{itemize}


\subsection{Calibration}
Before you begin your observations of the hydrogen clouds, you have to calibrate the telescope similarly to how you did in the previous lab. However, this time you have two options for calibrating: The offset frequency method and the off-source method. The offset frequency method can be used to calibrate the telescope while its "on-target" pointing at a hydrogen source. It relies on switching to a frequency far enough away from the hydrogen line so that the telescope only detects receiver noise and thermal background. The off-source method allows one to calibrate the telescope at the hydrogen frequency by pointing the telescope at a position far off-source to avoid hydrogen emissions. However, the off-source method does not account for accidentally picking up hydrogen emissions from some other source. 


\begin{steps}
	\item In the control panel, the positions of different Galactic longitudes along
	the equator are indicated by Gxxx (where xxx is in degrees). If you see
	longitude of 90 degrees and smaller, start at the longitude of 90
	degrees and work your way down to the smallest longitude you can
	observe down to zero of the center of our Galaxy (coincident with the
	source named Sgr A). If you cannot observe longitudes smaller than
	90 degrees observe one or two longitudes that are up in the sky. Note
	that galactic coordinates labeled in the SRT display can be pointed to
	by clicking on them – for the rest you have to estimate (or look up) the
	Az El of the desired longitude on the galactic plane. 
	
	\item Move to your first pointing and press Clear button to clear the spectrum accumulated by the SRT.
	
	\item  Set the frequency to 1420.4 4 and obtain the spectrum for the galactic longitudes visible in the ky by integrating for 10-20 seconds (or longer) along the same pointing
	
	\item After integration click on the spectrum window to get a detailed plot of
	the spectrum in a separate window. You will see emission flux as a
	function of frequency and VSLR. Estimate that maximum velocity of
	the clouds visible in the spectrum.
	
	\item Record the spectrum by first clicking on the spectrum in the viewer to
	bring up the detailed plot, then use Alt-PrintScreen to capture the plot
	in the clipboard and then press Ctrl-V to paste the clipboard in the
	Excel window. Be sure to maximize the window size to make your plot as legible as possible. 
	
	\item Proceed to the next Galactic longitude available for observation. Press Clear button to clear the spectrum before you make observation at
	each new longitude. Record longitude and spectrum in the Excel file
	along with your estimate of the maximum velocity for each pointing.
\end{steps}


















