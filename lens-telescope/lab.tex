\chapter{Bending light to see into space}

Optical telescopes are one way that astronomers use to better observe the cosmos. In this lab, you will build up your understanding about how telescopes help us do this.

\section{Learning Goals}

\begin{itemize}
	\item Learn how light behaves when traveling between mediums.
	
	\item Create an image with a lens.
	
	\item Configure a refracting telescope and explain how it helps for observing the sky.
\end{itemize}

\section{But first! Observing with Stone Edge Observatory}

Throughout the quarter, you will be taking data (images) with a robotic telescope that is located in Sonoma, California. This week, you will select a target and queue some images to be taken, at \url{queue.stoneedgeobservatory.com}. Please choose from one of the targets in Table\ \ref{lt:tab:targets}. The camera is a monochrome camera, broadly sensitive to all visible wavelengths ($\sim 400$--$700\:$nm). For now, let's collect light in all of those wavelengths by selecting the ``Clear'' filter, as well as leaving ``Dark'' checked.

\begin{table}
	\begin{tabular}{p{0.6in}p{.8in}p{.9in}p{0.6in}p{0.9in}p{2in}}
		target & RA & Dec & V-band & Suggested & Notes \\
		& (hh mm ss) & ($\pm$ dd mm ss) & magnitude & exposure time (s) & \\ \hline
		51 Pegasi & 22 57 27.980 & +20 46 07.7822 & 5.46 & 15 & first Main Sequence star found to host an exoplanet \\
		Altair & 19 50 46.999 & +08 52 05.9563 & 0.76 & 5 & rapidly rotating star
	\end{tabular}
	\caption{List of targets for you to choose from.}\label{lt:tab:targets}
\end{table}

So that the queue is not overwhelmed, please limit your group's observations to one queue submission per student per week.

\begin{itemize}

	\item Target, RA (hh mm ss), Dec (±dd mm ss), V-band magnitude, Suggested exposure time in seconds, Notes about the star
	
	\item 51 Pegasi , RA: 22 57 27.9804167474, Dec: +20 46 07.782240714, V = 5.46, 15 s, first Main Sequence star found to host an exoplanet
	
	\item Altair, RA: 19 50 46.99855, Dec: +08 52 05.9563, V = 0.76, 5 s, rapidly rotating star
	
	\item Betelgeuse, RA: 05 55 10.30536, Dec: +07 24 25.4304, V = 0.42, 5 s, red supergiant star and one of the largest stars visible with the naked eye

	\item Deneb, RA: 20 41 25.91514, Dec: +45 16 49.2197, V = 1.25, 10 s, forms the Summer Triangle with Altair and Vega 
	
	\item Kelt-9, RA: 20 31 26.3534038246, Dec: +39 56 19.765154336, V = 7.56, 30 s, hosts the hottest known gas giant exoplanet
	
	\item Epsilon Eridani, RA: 03 32 55.8449634, Dec: -09 27 29.731165, V = 3.73, 15 s, one of the nearest Sun-like stars with a planet and very young
	
	\item KIC 8462852 (aka Tabby's Star), RA: 20 06 15.4527143158. Dec: +44 27 24.791354432, V = 12.01, 90 s, has very irregular changes in brightness (ask Prof. Fabrycky about what causes the dips!)
	
	\item Mira, RA: 02 19 20.79210, Dec: -02 58 39.4956, V = 6.53, 25 s, variable binary system with a red giant and a white dwarf
	
	\item Polaris, RA: 02 31 49.09456,  Dec: +89 15 50.7923, V = 2.02, 10 s, lines up almost exactly with Earth's rotational axis to the north and is often called the North
	
	\item Sirius, RA: 06 45 08.91728, Dec: -16 42 58.0171, V = -1.46, 2 s, brightest star in the sky
	
	\item UY Scuti, RA: 18 27 36.5286196699, Dec: -12 27 58.893326502, V = 11.20, 90 s, largest known star by radius
	
	\item Vega, RA: 18 36 56.33635, Dec: +38 47 01.2802, V= 0.03, 5 s, one of the most well studied stars in the sky and has been used as a reference for calibrating photometric brightness scale for stars

\end{itemize}

\section{How do we use materials to bend light?}

We use lenses all the time to shape the path that light takes, either with eyeglasses or with optical telescopes. These activities are intended to help you understand how we use materials to bend light.

\begin{steps}
	\item Find the laser and a trapezoidal prism in your kit. Set the prism on a white sheet of paper and adjust the laser so that it is shining into the prism from the side. Play with the angle at which the beam strikes the prism and observe what happens to its direction as it enters the prism (ignore the reflections or transmission through the other side, for now). How does the path of the laser beam change when it enters the prism? \textbf{Record your observations. Be specific about any patterns you notice.}
	
	\item It's hard to see the light in the prism, so let's use a simulation. Go to \url{https://phet.colorado.edu/en/simulation/bending-light} and select the play button to launch the simulation. Select the leftmost ``Intro'' box. This is a side view of the interface between two materials, currently air on the top half of the screen and water on the bottom. A laser is positioned above the interface.
	
	\item Play with the controls on this screen until you have an idea of how to move the laser, turn it on and off, and adjust the materials on the top and bottom. To reset the simulation, select the orange circular button on the lower right.
	
	\item Observe what happens to the laser beam when you change the a) angle, b) type of material on top and bottom, and c) indices of refraction. In each case, does the beam get deflected from its original straight-line path more, less, or the same amount? \textbf{Record your observations for your lab report in a table format.}

	\item Open the second screen at the bottom of the sim. Lenses are wider in the middle and thinner at the top and bottom, and we can model this with a circular prism. Drag a circle up and shine a ray through it. Set the laser to output several parallel rays and aim it so they hit the middle of the circle. \textbf{Record your observations of what the rays do when the exit the other side of the prism.}
	
	\item Go back to your physical ray box and optics set. Adjust the ray box to emit 5 parallel rays. From the Pasco Optics Set, set up the convex lens (the one that is thinner on the ends and thicker in the middle) so that the rays are hitting it from the side. Notice where the rays go after they go through the lens. \textbf{Record your observations.}
\end{steps}

This setup of parallel rays is convenient for seeing precisely what the lens does. It also happens to be the situation when we observe things that are very far away compared to length scales of the lens. Consider two stakes driven into the ground next to each other, both perpendicular to the ground (and thus pointing directly at the center of the Earth). Since the Earth is a sphere, those stakes can't actually be both pointing directly toward the center of the Earth and also parallel to each other. For the former to be true, they must be angled slightly away from each other. But since the distance between them is so short compared to the distance away from the Earth's center, they are effectively parallel. \textit{This is the same with light arriving from distant objects like stars.}

\begin{steps}
	\item Let's try putting some parallel rays from a distant object on a lens and see what happens where those rays intersect. Take one of the round lenses that are mounted in a black plastic frame and shine those parallel rays from the light box on it to find the distance where the rays intersect with each other. Hold a sheet of paper or hand in the path of the rays so you can see them.
	
	\item Select a distant bright object with sharply contrasting edges. A good choice is the florescent ceiling lights in the hallway outside the lab. Hold the lens between the object and a white sheet of paper (or the floor if you are using the hallway ceiling lights). The white sheet of paper should be placed about the same distance away as the intersection distance from the previous step. \textbf{Record your observations.}
\end{steps}

If the object is a long distance away from the lens, compared to this intersection distance, then this intersection distance is called the focal length of the lens. The focal length is a property of the lens, based on its material and curvature. If an object is a long distance away compared to the focal length, then its image is formed at the focal length (if the object is closer, the image is formed further away than the focal length).

This principle works in reverse too --- if an object is placed at the focal length of the lens, the rays come out parallel on the other side. The image is effectively formed an infinite distance away.

\begin{steps}
	\item Design and conduct an experiment to measure the focal length of the lens you just used. Decide as a group how to measure, and how to estimate an uncertainty for your measurement. See Appendix\ \ref{cha:uncertainty} for detailed information about estimating uncertainty. \textbf{Record a sketch of your setup, a description of your procedure for gathering and analyzing your data, and the data itself, including your value of focal length with its uncertainty.}

	\item Compare: how does this focal length compare to the focal length that is printed on the lens holder? See Appendix\ \ref{unc:sec:comparing} for how to compare two values, taking into account their uncertainties. \textbf{Record this comparison calculation and what you conclude about how close they are. Is the printed value correct?}
\end{steps}

This lens setup is great for producing images, for example to record onto photographic film or a digital camera's image sensor. It's less good for looking through the lens to magnify and gather more light the way we want to with a telescope. For that, we'll need at least two lenses.

\section{Your first telescope}

Telescopes come in many different configurations. Here you'll construct one that is simple by comparison, a refracting telescope, using just 2 lenses.

\begin{steps}
	\item Here's the principle for building this telescope: the image created by the first lens, called the objective lens, is the object for the second lens, which is called the eyepiece lens. The thing we are wanting to look at (the object for the objective lens) is far away. We want the image created by the eyepiece lens to be an infinite distance away on the near side (just trust me on this). \textbf{Given these design goals and the information about lenses above, where should the two lenses be positioned with respect to each other? Sketch your proposal, labeling each lens and drawing the focal lengths of each lens.}
	
	\item Using the optical bench, construct and test your telescope. Look at a distant (across the room) object through it. If you don't get a clean image of it by looking through the eyepiece lens, iterate on your design until you get it.
	
	\item Find the magnification of your telescope. Hint: if you have two identical objects, how close does one need to be to look the same size/distance as the one you see through the telescope? Experimentally determine this magnification factor (1 is no magnification, 2 means it looks twice as close, 0.5 means that it looks twice as far away, etc), and estimate the uncertainty of your magnification. \textbf{Record this.}
	
	\item Magnification should be related to the properties of the two lenses. Make up a formula that relates the focal lengths of your lenses to the magnification of your telescope. You may need to switch the lenses around or use different ones to test your formula. \textbf{Record this.}
\end{steps}

\section{Report checklist and grading}

Each item below is worth 10 points, and there is an additional 10 points for attendance and participation. See Appendix\ \ref{cha:lab-report-format} for guidance on writing the report and formatting tables and graphs.

\begin{itemize}
	\item Detailed observations from Steps 1--8.
	
	\item Sketch of your setup and procedure for finding the focal length in Step 9, as well as value, with uncertainty, of the focal length.
	
	\item Comparison of your measured focal length to the manufacturer's stated focal length, from Step 10.
	
	\item Detailed sketch of your working telescope design, from Steps 11--12.
	
	\item Experimental determination of the magnification of your telescope, with uncertainty.
	
	\item Formula relating the focal lengths to the magnification.
	
	\item Discuss the findings and reflect deeply on the quality and importance of the findings. This can be both in the frame of a scientist conducting the experiment (``What did the experiment tell us about the world?'') and in the frame of a student (``What skills or mindsets did I learn?'').
	
\end{itemize}