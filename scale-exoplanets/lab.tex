\chapter{Scale of the solar system and local stellar environment}

%todo for proxima group, add the voyager probes and the heliopause (with info about probes, when sent, etc)

\begin{quotation}
	\textit{Space is big. You just won't believe how vastly, hugely, mind-bogglingly big it is. I mean, you may think it's a long way down the road to the chemist's, but that's just peanuts to space.} \sourceatright{Douglas Adams, The Hitchhiker's Guide to the Galaxy}
\end{quotation}

Why can't we just see planets orbiting other stars with normal observational techniques like looking through larger and larger telescopes? Through this lab, we hope you gain a felt sense appreciation for the scale of star systems and the distance between the Sun and our closest stellar neighbor, and see why we need to use special techniques to detect exoplanets.

\section{Your current intuition}\label{se:sec:intuition}

In order to gauge your current sense of where things are in relation to each other, each member of your group will first make three different small, qualitative scale models. \textbf{For this section, do not use a book, or the Internet, other people, or any other resource, to guide your efforts.} This will help you see where your current intuition lies. And when you compare to others, do not change your guess --- it is expected that you might not have an accurate conception yet.

\begin{enumerate}
	\item Individually, without looking at your groupmates' work, take a blank sheet of paper and draw a horizontal line across the page. On the left end of it, place a dot and mark it ``Sun''. On the right end, place a dot and mark it ``Neptune''. Now place and label a dot for your best guess orbital distance for each of the seven other planets orbiting the Sun.

	\item Next, make a similar scale model, this time placing the following objects on it: the Sun, Neptune, our nearest star Proxima Centauri, and its planet, Proxima Centauri b.
	
	\item Finally, draw the Sun and all 8 planets with your best guess of their relative sizes --- you should end up with 9 circles. The distances do not need to be scaled as well for this estimate.

	\item Compare your drawings with your groupmates. Write down any surprises or big differences that you had for your report.
\end{enumerate}

\section{Making an accurate scale model}\label{se:sec:model}

Now that you have your current sense of it, you will make an accurate scale model of the three scales you made above. Bigger is better, so you will use the hallway that extends on the second floor from the south end of Kersten, across the skywalk, through Eckhardt, and into the Accelerator Building.

\textbf{Available equipment:} measuring wheel, paper, masking or label tape, scissors, markers

Your group will be assigned one of the 3 scale models described in the previous section to create in the hallway. After you create it, you will give a short ($\sim 5\:$min) presentation to the class, describing what you found and what your impression is, and walking the class through the scale model down the hallway.

\subsection{Tips}
\begin{itemize}
	\item To make a scale model, you will need to gather the size or distance information for the objects in your model using any resource you'd like, then divide each of those by the same number to create your scale. You can use a spreadsheet to make this task easier.
	
	\item For the two distance scale models, you will need to measure the distance you have to work with in the hallway. Ensure that you can see from one end to the other. Prop open the skywalk doors if needed. You should place some kind of upright sign, perhaps taped down, at each spot where an object is in your model.
	
	\item For the size scale model, since all the objects are mostly spherical, you can create your model by cutting out circles of paper. For the larger objects, feel free to tape sheets of paper together to make a bigger circle.
\end{itemize}

\section{Appreciating distances}\label{se:sec:calc}

Answer the following questions:
\begin{enumerate}
	\item Using a nominal speed of a car on Earth, how long would it take to drive:
	\begin{enumerate}
		\item once around the equator?
		\item from the Earth to the Sun?
		\item from the Sun to Neptune?
		\item from the Sun to Proxima Centauri?
	\end{enumerate}
	\item If you wanted to travel each of these distances in 1 year, how fast would you need to go in each case?
	\item For the longest distance, how does that speed compare to the speed of light, which is the fastest anything can go?
\end{enumerate}

\section{Report checklist}

Include the following in your lab report. See Appendix~\ref{cha:lab-report-format} for formatting details.

\begin{itemize}
	\item Your intuition estimates from Section~\ref{se:sec:intuition} and your reflection of how they compare to your group mates'.
	\item A table of the distances/sizes and scaled version that you created in Section~\ref{se:sec:model}.
	\item Worked solutions to the questions in Section~\ref{se:sec:calc}.
	\item A 100--200 word reflection on the assignment --- was there anything that surprised you? How do you see your place in the universe, given the scale of the solar system and how far away even the nearest star is?
\end{itemize}
