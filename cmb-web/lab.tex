\chapter{Modeling the Cosmic Microwave Background}

%todo what do we do about the "ANSWER" button in the sim?

In this lab, you will analyze the power spectrum of the cosmic microwave background (CMB). It is assumed that you have already heard of it. If you don't know what it is yet, search online and learn a little about it. Once you're back, continue with this lab.

\begin{steps}
	\item Go to the CMB Analyzer simulation at \url{https://map.gsfc.nasa.gov/resources/camb_tool/index.html}.
\end{steps}

This interactive simulation allows you to model the CMB angular power spectrum. You can change the different parameters that go into the model and see how it affects the spectrum. You can even change the parameters until your model spectrum matches the experimental spectrum. This would give you confidence that your chosen parameters are correct.

\begin{steps}
	\item Play with the sliders and read through the text that appears when you hover on a slider. Move a few sliders around and observe how they affect the power spectrum.
	
	\item To understand the CMB and the power spectrum, read through the ``Parameters of Cosmology'' here: \url{https://map.gsfc.nasa.gov/mission/sgoals_parameters.html}. There are 7 pages to this document, and you can navigate using the ``next page'' link in the lower right of the text. While you learn about each parameter, go to the CMB Analyzer and vary that parameter to become more familiar with it.
	
	\item For each parameter that can be varied in the model, describe what the parameter means for the universe and how changing that variable changes the power spectrum. \textbf{Record this for your report.}
\end{steps}
	
For complicated models like this, scientists learn about the universe, for example the proportion of matter, dark matter, and dark energy, by changing the free parameters in their model to see what best fits the observed data. If they trust that their model contains the right physics, then the parameters that fit best are likely to be correct.

\begin{steps}
	\item Move the sliders to find the parameter values that make the model fit the experimental data best. \textbf{Take a screenshot of the best fit and record the values.}
\end{steps}

\section{Testing the Big Bang Theory}

Here is a hypothetical observation that is \textit{not} predicted by the Big Bang theory:\footnote{This discussion is from Bennett, Donahue, Schneider, Voit, The Cosmic Perspective, 9th ed. (2020)}

\begin{quote}
	``evidence for an increase in the cosmic microwave background temperature with time''
\end{quote}

\begin{steps}
	\item Imagine what would happen if it were actually observed --- whether it could be explained with the existing Big Bang theory, could be explained with a revision to the Big Bang theory, or would force us to abandon the Big Bang theory. \textbf{Write down your team's reasoning.}

    Note: you may want to assume these four roles for your discussion: \textit{Scribe} --- takes notes on the group's activities; \textit{Proposer} --- suggests tentative explanations to the group; \textit{Skeptic} --- points out weaknesses in proposed explanations; \textit{Moderator} --- leads group discussion and makes sure everyone contributes and no one is dominating the discussion.
\end{steps}

\section{Report checklist and grading}

Each item below is worth 10 points. Every item except the last one can be identical between lab group members.

\begin{enumerate}
	\item Parameter descriptions and effect on power spectrum (Step 4)
	
	\item Best fit parameters and screenshot of best fit (Step 5)
	
	\item Conclusion and reasoning for effects of hypothetical observation on Big Bang theory (Step 6)
	
	\item A 100--200 word reflection on group dynamics and feedback on the lab manual. Address the following topics: who did what in the lab, how did you work together, what successes and challenges in group functioning did you have, and what would you keep and change about the lab write-up?
\end{enumerate}