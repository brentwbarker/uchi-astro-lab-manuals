\documentclass{article}
\usepackage{url}
\usepackage{hyperref}

\title{SRT In-Class Technician Guide}
\date{Updated Winter 2021}

\begin{document}
	\maketitle
	
\section{other docs to reference}
	
	In addition to this document, ensure that you have access to the lab manual, SRT manual, and lab syllabus (\url{https://docs.google.com/document/d/1atZc3x0wssQKxKspgm-VopQvrsbDo4wab1v80nA2zj8/edit?usp=sharing}).
\section{Your role}
	
	During an in-person class, the students would be operating the computer themselves. In this remote environment, you are being that computer. You are in charge of starting the SRT control software, connecting to the camera, and handling any bugs or software issues that come up. The students should tell you what commands to enter and what buttons to press in the SRT control. If students are confused, you can refer them to relevant parts of the lab manual or have them ask a TA. If you want to, you can also act as a teacher, guiding them through the lab without telling them answers or doing things for them.
	
\section{In case you need help}

Don't panic! I don't think you can harm the telescope by operating it through the control software. If you get stuck or confused, you can attempt to contact Brent Barker through email (bbarker@uchicago.edu), text/phone (724-422-2611), or the A+A Slack.

\section{Setting things up}

\begin{enumerate}
	\item Connect to the campus VPN. You can visit cvpn.uchicago.edu to download the VPN client, and then in the VPN client, connect to that same URL to start your VPN. This routes all your internet traffic through the UChicago network.

	\item Remote desktop to SRT computer. The IP address for the computer is 128.135.156.222, username is ``PSCD Student'' and password is ``student''. If you are on Linux, a convenient remote desktop client is ``remmina''.
	
	\item In the remote desktop, open Firefox and navigate to 10.120.156.47 (or click the ``SRT Camera'' bookmark). Log in to the Amcrest camera with username ``justaperson'' and password ``hubblewasnotagiraffe!''.
	
	\item On the desktop, double-click on ``srt'' to load the SRT control software.
	
	\item Move and resize the windows so that both the camera and control software are visible at the same time.
\end{enumerate}
	
\section{connect to classroom}

Join the class through Zoom or Gather.town. Links in the lab syllabus. In Zoom, go to your breakout room and share your screen. In Gather.town, go to your place in the room and share your screen. Ask the TA if you don't know where to go.

\section{Closing at end of session}

\begin{enumerate}
	\item Ensure that the telescope is stowed.
	
	\item close the SRT software and camera webpage.
	
	\item Log out of the SRT control computer by clicking on the Windows icon on the lower-left, moving the pointer up to ``PSCD Student'' (the icon above Documents), left-clicking, and selecting ``Sign Out''.
	
	\item Disconnect from the campus VPN if you're done using the campus network.
\end{enumerate}
	
\section{tips for operating telescope}

\begin{itemize}
	\item The SRT software does not automatically redraw things if they get blocked by other windows. So if you minimize the window or move another window over it, you will need to wait a few seconds for it to redraw everything. Try to avoid doing that.
	
	\item If there is a communication error, try clicking Stow and watching the camera to see if it obeys. You can also try exiting out of the program (clicking the X) and restarting it. It is safe to do this even if the telescope is not stowed.
	
	\item There are minimum elevation and azimuth set. If you enter a very small value and nothing moves, try a larger one. ``20 30'' is clearly within bounds.
\end{itemize}
	
\end{document}