\chapter{Discovering the Hydrogen Atom}

\section{Learning Goals}

% these are statements of what we want the student to be able to do at the end of the activity
\begin{itemize}
	\item Critically evaluate model of physical phenomena and understand how these models are formulated.
	
	\item Develop methods for testing and working with objects which cannot be directly observed. 
	
	\item Understand the atomic model and why it differs from what classical physics predicts.
\end{itemize}

\section{Lab Team Roles}

Decide which team members will hold each role this week: facilitator, scribe, technician, skeptic. If there are three members, consider having the skeptic double with another role. Consider taking on a role you are less comfortable with, to gain experience and more comfort in that role.

Additionally, if you are finding the lab roles more restrictive than helpful, you can decide to co-hold some or all roles, or thinking of them more like functions that every team needs to carry out, and then reflecting on how the team executed each function.

\section{Model Evaluation and Construction}
When thinking of an atom, we usually tend to imagine a series of electrons orbiting around a nucleus in nice circular orbits --- this is the planetary model of the atom.While at first this model might seem unremarkable, the reality is that the physics that describes it is a lot weirder than you think! 
% this should be the goal of this particular section. Some of this is about the whole lab, so it can go above the Learning Goals section. Other parts are an introduction to this section, which should go right above the Goal section. The Goal section should be a succinct description of what the task of this section is and what is produced at the end.

\subsection{Goal} 
Analyze the planetary model using classical physics concepts, determine its shortcomings, and begin to develop your own model of the atom. 

\begin{steps} 
	\item First, let's take the classical planetary model of the atom at face value. Using only classical physics concepts, find what's wrong with this model. Take the system of the electron orbiting the nucleus to be equivalent to that of a satellite orbiting the Earth. 
	\begin{itemize} 
		\item Think first about what happens to a satellite orbiting Earth as it interacts with the atmosphere. What forces are acting on it? What happens to it over time?
		
		\item Consider that changing the orbital radius of a body requires a corresponding change in energy. Think about the changes in kinetic and potential energy as the satellite moves closer or farther from the Earth. 
	\end{itemize}

	\item Once your group has come up with an answer, describe what happens to the orbiting object in terms of its energy and its position over time. What implications does this have when you apply it to the atom? \textbf{Write down your explanation}
	
	\item After identifying the problem with the model, develop 2 or 3 ideas for a new model which solves the issues you found. \textbf{Forget any previous knowledge of the atom. Think outside the box. Write down your ideas and describe how they solve the problem.} 
	\begin{itemize}
		\item These models don't have to be too complicated, just try thinking of different ways in which the electron might move or behave in relation to the nucleus.
	\end{itemize} 
\end{steps}

\section{Testing and Observation: How does an atom absorb and release energy?} 

Now that you have a rough model of the atom, lets supplement that with some observations to see if you can make it more robust. The difficulty with trying to learn about the structure of the atom is that we have no way of directly observing it. However, we are able to interact with it and observe the results.

\subsubsection{Goal}
Fire photons at a hydrogen atom, determine it interacts with light energy, and modify your model based on your observations.

\subsubsection{Available Equipment}

\begin{itemize}
	\item PHET Models of the Hydrogen Atom Lab: \url{https://phet.colorado.edu/en/simulation/legacy/hydrogen-atom}
\end{itemize}

\subsubsection{Rubrics to be assessed}

B5, B7, B9, C4, C5

\subsubsection{Steps}

\begin{steps} 
	\item Once in the simulation, make sure to turn on the spectrometer, and familiarize yourself with the tools available. Available to you are photon gun which can be set to white or monochromatic (single color) light, a spectrometer which you can take snapshots of, and a speed toggle.
	
	\item This simulation is meant to demonstrate how an atom absorbs and emits energy in the form of photons of light. Based on this, what predictions, if any, does your model make? \textbf{Record your model's predictions.}
	
	\item Fire white light at the atom. What patterns do you observe? Record any qualitative observations and describe the pattern. 
	
	\item Turn off the light, reset the spectrometer, and switch over to monochromatic light. Fire light of several different wavelengths (colors) at the atom. \textbf{Record your observations.}
	
	\item Now, for each of the wavelengths listed, record what happens when photons at that wavelength are fired at the atom. Describe any patterns you observe and include screenshots of the spectrometer. (Pay attention to the order in which the photons are emitted)
	\begin{itemize}
		\item Wavelengths to test: 122nm, 103nm, 97nm, 95nm, 94nm 
		
		\item \textbf{Important: Sometimes the simulation might freeze and the atom will appear to no longer emit photons. If that happens, switch over to the white light setting, wait until the atom starts emitting again, switch over to monochromatic light, reset the spectrometer, and continue.}
	\end{itemize}
	
	\item One thing you might have noticed is that for certain wavelengths, the atom might emit several different colors of light. Why might this be happening? Think about it in terms of the energy being absorbed and emitted by the atom. Does it make sense for the atom to be emitting photons at higher wavelengths than the ones absorbed? Remember that energy is inversely proportional to the wavelength of the photon (as one increases the other decreases). Why does the atom absorb at some wavelengths and not others? \textbf{Record your answers to these questions.} 
	
	\item Once you believe you have found a pattern, modify your model of the atom so that it is able to explain these new observations. Record any changes you made and describe how these explain the observed patterns.
	\begin{itemize}
		\item Think about where the energy goes in the atom once the photon is absorbed.
		
		\item Does your atom somehow change following an increase in energy? It might help to think back to the classical orbit analogy used in the previous experiment.   
		
		\item How does your model account for the fact that only very specific amounts of energy are absorbed and emitted? 
	\end{itemize}
\end{steps}

\section{Model Evaluation}

\subsection{The Building Blocks of Everything}
 The word atom finds its roots in the Greek word \textit{atomos} which means indivisible and is believed to have first been used by Democritus to refer to the indivisible spheres which he believed to be the building blocks of our world. Today, we have a far better understanding of the atom and its structure. However, this is information we take for granted. For millennia, there was no clear answer as to what the smallest unit of everything was, and only relatively recently did we develop tools which allowed us to study them more closely. The first comprehensive atomic model was developed in 1803 by John Dalton, who thought of them as solid spheres which changed depending on the element they made up. Then, in 1904, JJ Thompson developed the ``Plum Pudding'' model, which proposed that the atom as made up of electrons floating within a cloud of positive charge. Then in 1911 Ernst Rutherford proposed the nuclear model, where the positive charge was concentrated in the center of the atom. Two years later, Niels Bohr improved upon Thompson's model, suggesting that the orbits were fixed. Finally, Erwin Schrodinger proposed in 1926 the quantum model, where electrons exist around the nucleus in a ``cloud of probability''. You will now have a chance to interact with each of these models.

\subsection{Goal} 
 Evaluate each model and try to develop an understanding for the reasoning behind it. You will also quantify the energy interactions in an atom, and see how the principle of conservation of energy is maintained. 

\subsection{Available Equipment}

\begin{itemize}
	\item PHET Models of the Hydrogen Atom Lab: \url{https://phet.colorado.edu/en/simulation/legacy/hydrogen-atom}
\end{itemize}

\subsection{Steps}

\begin{steps}
	\item Make sure the simulation is in ``Prediction'' mode. On the left, you will see a list of all the atomic models organized from ``classical'' to ``quantum''.
	
	\item For each model:
	\begin{itemize}
		\item Provide a qualitative description of its structure and behavior. How does it absorb energy? Where does it go once it's absorbed? Are there any structural changes? etc. 
	
		\item Along with each description, provide an explanation of the possible reasoning behind each model. What problems does each model address? Why would each model be a good guess as to the structure of an atom? Likewise describe the limitations of each model, if anything. What does it fail to describe or account for?
	
		\item Now compare the model you came up with to some of the other models. Does it resemble any of the other models? Does it improve on some of the shortcomings of the other models? What did you take into account that they did not? Was there anything missing from your model? What did you fail to consider? \textbf{Write down your observations.}
	
		\item \textbf{Along with each description, include a screenshot of the spectrometer for the model.} 
	\end{itemize}
	
	\item In the previous experiment, you saw how atoms can only absorb and emit a very specific amount of energy. Using the Bohr model, calculate the energy being emitted by the atom when it absorbs light with wavelength $\lambda = 94$nm. 
	\begin{itemize}
		\item Use the equation $\mathit{E} = \mathit{h}\nu$, where $\mathit{E}$ is energy (SI unit is the joule (J)), $\mathit{h}$ is the Planck constant equal to $6.62 \times 10^{-34}\:$J$\:$s, and $\nu$ (Greek letter pronounced like ``new'') is the frequency of the emitted photon (SI unit is hertz (Hz), or s$^{-1}$).
		
		\item To convert from wavelength to frequency use the equation $\nu = \dfrac{c}{\lambda}$. 
		
		\item Make sure to include uncertainties in your calculations.
		
		\item How does the total emitted energy compare to the energy absorbed? Are they equal?
	\end{itemize}
	
	\item Sometimes, you might see that the atom emits light at 656nm (red light). Why might the atom emit but not absorb light at that same wavelength? \textbf{Write down your answers}
\end{steps}

\section{Group dynamics}

\begin{steps}
	\item Write a 100--200 word reflection on group dynamics and feedback on the lab manual. Address the following topics: who did what in the lab, how did you work together, what successes and challenges in group functioning did you have, and what would you keep and change about the lab write-up?
	
	\item Write a paragraph reporting back from each of the four roles: facilitator, scribe, technician, skeptic. Where did you see each function happening during this lab, and where did you see gaps?
\end{steps}

\section{Report checklist and grading}

The lab grade consists of 3 points for each of seven scientific ability rubric rows (the 5 listed above, which apply just to that section, as well as F1 and F2, applied to the entire report), and 9 points for providing evidence in the lab report of completing all steps of the lab, including answering every question, for a total of 30 points.