\chapter{Discovering the Hydrogen Atom}

\section{Learning Goals}

\begin{itemize}
	\item Learn to identify errors in physical models and modify them through inference and observations of physical phenomena
	
	\item Develop methods for testing and working with objects which cannot be directly observed
	
	\item Gain an understanding of the properties of atomic structures and why these differ from what classical mechanics would predict
\end{itemize}

\section{Lab Team Roles}

Decide which team members will hold each role this week: facilitator, scribe, technician, skeptic. If there are three members, consider having the skeptic double with another role. Consider taking on a role you are less comfortable with, to gain experience and more comfort in that role.

Additionally, if you are finding the lab roles more restrictive than helpful, you can decide to co-hold some or all roles, or thinking of them more like functions that every team needs to carry out, and then reflecting on how the team executed each function.

\section{Model Evaluation and Construction} 

\subsection{Goal} 
When thinking of an atom, we usually tend to imagine a series of electrons orbiting around a nucleus in nice circular orbits. While at first this model might seem unremarkable, the reality is that the physics that describes it is a lot weirder than you think! In order to see why this is, lets see if we can reconstruct the model of the atom qualitatively first.

%\susubbsection{Available Equipment:}

%\subsubsection{Rubric Rows to Focus On:}

\begin{steps}
	\item First, lets take the classical planetary model of the atom at face value. Using only classical physics concepts, try and find whats wrong with this model. 
	\begin{itemize} 
		\item Think first about what happens to satellites orbiting Earth as they interact with the atmosphere. What forces are acting on it? What happens to it over time?
	\end{itemize}
	\item Once your group has come up with an answer, describe what happens to the orbiting object in terms of its energy and its position over time. What implications does this have when you apply it to the atom?
	
	\item After identifying the problem with the model, try and come up with several ideas for a new model which solves the issues you found.
	\begin{itemize}
		\item These models don't have to be too complicated, just try thinking of different way in which the electron might move or behave in relation to the nucleus.
	\end{itemize}
\end{steps}

\section{Observation Experiment: How does an atom absorb and release energy?}

\subsubsection{Goal:}
Now that you have a rough model of the atom, lets supplement that with some observations to see if you can come create a more robust mode. The difficulty with trying to learn about the structure of the atom is that you have no way of directly observing it. However, you are able to interact with it and observe the results. In this experiment, you will be firing photons at an atom and you will attempt to modify your model of the atom based on your observations.
\subsubsection{Available Equipment}

\begin{itemize}
	\item The PHET Lasers simulation: \url{https://phet.colorado.edu/en/simulation/legacy/lasers}
\end{itemize} 

\subsubsection{Rubrics to be assessed in this experiment}

\subsubsection{Steps}

\begin{steps}
	\item When you enter the simulation, on the right-hand panel select the option for "three" under the "Energy Levels" heading.
	
	\item Take a moment to familiarize yourself with the different parameters you can control. In particular focus on the lamp controls and the energy level controls. 
	
	\item In your group, try and find a pattern in the absorption and emission of photons.
	
	\item Once you believe you have found a pattern, formulate a hypothesis for how atoms absorb energy from photons and try to modify your model based on this hypothesis
	\begin{itemize}
		\item Think about where the energy goes in the atom once its absorbed
		
		\item Which particle in the atom do you think would have the most freedom?
		
		\item Remember last lab's discussion of gravitational potential energy. How would the atom or its individual components change if it absorbed more energy? 
	\end{itemize}
\end{steps}

\section{Testing Experiment: Black-boxing the Hydrogen Atom}

\subsection{Available Equipment}

\begin{itemize}
	\item PHET Models of the Hydrogen Atom Lab: \url{https://phet.colorado.edu/en/simulation/legacy/hydrogen-atom}
\end{itemize}

\subsection{Rubrics to be assessed during this experiment}

\subsection{Goal}
Now that you have worked to develop your own theoretical model of the atom, you will see how they compare to real life theoretical models proposed for the atom. You will be given a black-box, behind which is a hydrogen atom. Since it can't be directly observed, you will try and determine which model of the atom is correct by expanding on the experiment you carried out in the previous simulation. 

\subsection{Steps}

\begin{steps}
	\item At first, leave the simulation in "Experiment Mode" and familiarize yourself with the available controls. 
	
	\item Notice the spectrometer which allows you to see the wavelengths of light emitted by the atom. With your group, try and determine a pattern in the emission spectrum of the atom. 
	
	\item Describe the pattern you observe. Does it fit with any of your earlier observations or predictions? In the previous experiment, you saw how atoms can only absorb and emit a very specific amount of energy depending on its energy levels. See if you can predict how many energy levels the hydrogen atom has.
	\begin{itemize}
		\item Remember that each color or wavelength of light is associated with a specific energy. Also keep in mind that, as the atom transitions between energy levels which are "close together" it will emit lower energy photons. 
		
		\item It might also help to keep track of the order in which photons are emitted.
	\end{itemize}

	\item Once you have an estimate for the number of energy levels, make a table of energy levels with their associated transition energies from that level to the "base" level. Use the equation ----- to calculate the energies from the wavelengths %Add explanation of equation
	
	\item Now switch over to the "Prediction" mode. Go through each model and compare the emission spectrum to the one you observed. 
	
	\item Which model or models predict the emission spectrum observed? Was your estimate of energy levels correct? 
\end{steps}

\section{Report checklist and grading}
