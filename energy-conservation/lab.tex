\chapter{Energy Conservation}

\section{Learning goals}

\begin{itemize}
	\item Use energy conservation to discover relationships between kinematics and types of energy.
	
	\item Use energy conservation to solve a problem.
	
	\item Identify assumptions made during an experiment and how they might affect the results.
	
	\item Take video and analyze motion using video tracking software.
\end{itemize}

\section{Lab Team Roles}

Decide which team members will hold each role this week: facilitator, scribe, technician, skeptic. If there are three members, consider having the skeptic double with another role. Consider taking on a role you are less comfortable with, to gain experience and more comfort in that role.

Additionally, if you are finding the lab roles more restrictive than helpful, you can decide to co-hold some or all roles, or thinking of them more like functions that every team needs to carry out, and then reflecting on how the team executed each function.

\section{Energy}

Energy can be defined in several different ways. One way that is useful for this lab is that energy is the ability to do work --- that is, the ability a system has to change its environment. One way it has the ability to do this is by nature of its speed. If an object is traveling relative to its surroundings, it can run into things and affect them. This type of energy is called \textit{kinetic energy}. Another way an object has this ability to affect things is through its position. These ways are called \textit{potential energy}. Either being further away from massive objects (gravitational potential energy), being confined to stay near other atoms (chemical), or confined within an atomic nucleus (nuclear).

Energy can be transformed into different types, and is conserved throughout interactions. That is, accounting for all forms of energy transformations, the amount of energy before an interaction is always equal to the amount after. An intuitive analogy is given with mass conservation: if I have 2 kg of pebbles in one hand, and I move 1 kg of them to my other hand, then you can be confident that 1 kg remains behind, unless mass is being converted into other forms of energy.

The law of energy conservation is a powerful tool in physics, allowing for discoveries: for example, in radioactive beta decay, an electron is released from an atomic nucleus with a certain amount of kinetic energy. This energy varies, but physicists had found that all the energies after the decay did not add up to the energy beforehand. They theorized and later discovered that this was due to a then-undiscovered particle, the neutrino.

\subsection{Qualitative derivation of gravitational potential energy}

One type of energy was stated to be gravitational potential energy (call it $U_\mathrm{grav}$), the energy gained by being further away from other massive objects. Near the Earth's surface, you can investigate this on a qualitative level easily --- if you lift a book a little bit off the ground, $h_1$, and drop it on your foot, it falls back down and doesn't do much damage. If you lift it high off the ground, up to $h_2$, and drop it on your foot, it will hurt more and cause more damage. More ability to affect the environment, and thus more gravitational potential energy. So we have already discovered that with more height off the ground, there is more $U_\mathrm{grav}$. Further, if you set your foot on a chair of height $h_1$, and raise the book the same height $h_1$ above your foot as you did in the first case above, then you would receive the same damage. So raising the book from 0 to $h_1$ increased the book's $U_\mathrm{grav}$ the same amount as raising from $h_1$ to $2h_1$. One can infer from this that $U_\mathrm{grav}$ is proportional to the height raised, or $U_\mathrm{grav}\: \propto\: h$.

\section{Observation experiment: How is kinetic energy related to velocity?}

Now that you know how gravitational potential energy depends on height (at least near the Earth's surface), you might be tempted to study falling objects more. As objects fall, they pick up speed, and thus kinetic energy.

\subsubsection{Goal}

Use the law of energy conservation and the theory that $U_\mathrm{grav}\: \propto\: h$ to determine how kinetic energy $E_\mathrm{k}$ and velocity $v$ are related, using videos of falling objects that you take and analyze.

\subsubsection{Available equipment}

\begin{itemize}
	\item a small, dense, durable object to drop on the ground
	
	\item a video camera like one on a smartphone or your webcam
	
	\item the software Open Source Physics Tracker
\end{itemize}

\subsubsection{Rubrics to be assessed in this experiment}

B3, B7, B8, G4, G5

\subsubsection{Steps}

\begin{steps}
	\item Discuss your experimental setup and procedure, including analysis, with your group. How will you measure kinetic energy (directly or indirectly), how will you measure velocity, and what will you vary in order to change one of them to find a pattern?
	
	\item Record clearly the phenomenon you are investigating.
	
	\item Decide what physical quantities are to be measured and identify independent and dependent variables.
	
	\item Record a detailed description of what measurements you are making and how you will make them.
	
	\item Carry out your measurement plan, using the following guide for video analysis using Tracker.
\end{steps}

\subsubsection{Record the video}

\begin{steps}
	\item Find a good object to drop. It should be dense enough to not be slowed down significantly by air resistance.
	
	\item Using the camera on one of your group member's phones, record a video of the object falling.
	
	Here are some tips to get a quality video:
	\begin{itemize}
		\item Include an object of known length in the shot, at the same distance from the camera as the falling object. This gives a reference length, so that you can find how each camera pixel scales to the physical situation.
		
		\item Avoid parallax error by having the object be at about the same distance from the camera throughout the fall. Having the camera be farther away can help. Also, you can ensure that the top and the bottom of the fall are the same distance from the camera.
		
		\item Hold the camera steady.
	\end{itemize}
	
	\item Record that video and transfer the video to a computer that has Tracker installed.
\end{steps}

\subsubsection{Importing the data into Tracker}

In this part, you'll use Tracker to record the position of the object at each timestep. To do this, you'll need to tell it what direction ``down'' is in, what the scale of the image is, and when time $t=0$ is. Then you'll record the position of the object in every frame, and you'll plot velocity vs. time to find the velocity just before the object hit the floor.

\begin{steps}
	\item Open Tracker on a computer. You can install it on your own computer by visiting \url{https://physlets.org/tracker}.
	
	\item Optionally, watch this 3-minute tutorial on how to use Tracker: \url{https://www.youtube.com/watch?v=n4Eqy60yYUY}
	
	\item In Tracker, open your video.
	
	\item \textbf{Find frame when zero time is.} Move the slider below the video to the right to advance the frames until you find the first one in which the object is falling. Record that start frame number, which is found to the left of the slider bar in red.
	
	\item \textbf{Find the last relevant frame.} Keep moving the slider to the right until you find the last frame before the object hits the floor. Record that end frame number.
	
	\item To \textbf{tell Tracker about these frames}, click the 5th icon from the left on the toolbar above the video (``Clip settings'') and enter the start frame and end frame.
	
	\item \textbf{Tell Tracker how long things are.} In astronomy applications, this is known as the ``pixel scale''. Here we can just draw a line on the frame and tell Tracker how long that line is in real life. Click the 6th icon from the left (blue, with a ``10'') and select \texttt{New} $\rightarrow$ \texttt{Calibration Stick}. Shift-click to mark each end of your known length, and type in your known length, with units in the box that appears along the stick. Use ``m'' for meters.
	
	\item \textbf{Align the coordinate system.} In the toolbar, click the 7th icon from the left (magenta crossed lines). Click and drag the coordinate system's origin (the intersection of long lines) to the location of the object in the start frame.
	
	\item \textbf{Check to see if the camera was tilted.} Advance the video to see if the object moves along an axis. If it goes off at an angle, the camera was tilted compared to the direction of motion. In this case, rotate the coordinate system to align with the motion by clicking and dragging the small line that crosses one of the axes.
	
	\item \textbf{Tell Tracker where the object is in every frame.}
	\begin{enumerate}
		\item In the toolbar, click \texttt{Create} $\rightarrow$ \texttt{Point Mass}.
		\item Ensure the slider is at the start frame.
		\item Shift-click on the object. Notice that the frame advances to the next one automatically.
		\item Continue to shift-click to mark the object's position throughout the duration.
	\end{enumerate}
\end{steps}

\subsubsection{Analysis}

\begin{steps}
	\item \textbf{Ensure the correct axis is selected for analysis.} Look at the plot to the right of the video. If there is not a smooth-ish curved line, click on the axis label ``x (m)'' and choose instead ``y (m)''.
	
	\item To view the velocity graph, click on the vertical axis label on the plot to the right of the video and select vx or vy.
	
	\item Find the velocity that is relevant to your measurement plan and record it.
	
	\item Once you have a table of data that you can use to search for a pattern, copy it into SciDAVis and try different functions to see what pattern fits.
	
	\item Identify the pattern in words and record a mathematical expression (formula) that represents the pattern you found, with a discussion of how well your expression agrees with the data.
\end{steps}

\section{Application experiment: finding energy lost to drag}

\subsubsection{Goal}

Assuming that $U_\mathrm{grav} = Agh$ and $E_\mathrm{k} = \frac{A}{2} v^2$, where $g = 9.8\:$m$/$s$^2$ and $A$ is the same unknown constant in both equations, find the fraction of energy converted to thermal energy via air resistance, for a light falling object.

\subsubsection{Available equipment}

\begin{itemize}
	\item a light object to drop on the ground like gently crumpled paper or a coffee filter
	\item a video camera like one on a smartphone
	\item the software Open Source Physics Tracker
\end{itemize}

\subsubsection{Steps}

\begin{steps}
	\item Follow the steps listed in Rubric D (Table \ref{rubric:d}) to find the fraction of energy transformed into thermal energy. You do not need to do two independent methods as described in D5.
\end{steps}

\section{Group dynamics}

\begin{steps}
	\item Write a 100--200 word reflection on group dynamics and feedback on the lab manual. Address the following topics: who did what in the lab, how did you work together, what successes and challenges in group functioning did you have, and what would you keep and change about the lab write-up?
	
	\item Write a paragraph reporting back from each of the four roles: facilitator, scribe, technician, skeptic. Where did you see each function happening during this lab, and where did you see gaps?
\end{steps}

\section{Report checklist and grading}

The lab grade consists of 3 points for each of seven scientific ability rubric rows (the 5 listed above, which apply just to that section, as well as F1 and F2, applied to the entire report), and 9 points for providing evidence in the lab report of completing all steps of the lab, including answering every question, for a total of 30 points.