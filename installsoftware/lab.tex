\chapter{Making groups, installing the needed software}

In this lab, you will ensure that you have groups, and that you know how you will communicate with each other. You will also ensure that you have the software installed that you need to complete the labs this quarter. This software includes
\begin{enumerate}
	\item Zoom, for video collaboration and joining meetings
	\item DS9, for analyzing astronomical images
	\item Spreadsheet software (e.g. LibreOffice Calc or Microsoft Excel), for analyzing data and making plots
	\item Flash, for using a web-based simulation
\end{enumerate}

\section{Forming Groups}

\begin{steps}
 \item Fill out the group introductions spreadsheet found here: \url{https://docs.google.com/spreadsheets/d/1h3V1AqgsETIhE1c0Z7OZxpSnK9WcJQotNosq4jZkSds/edit?usp=sharing}, including taking the DOPE Bird Personality Quiz, linked to in the spreadsheet.
 
 \item Contact others in your lab section to form groups. Consider what kind of Bird Types you think would add to your group's effectiveness, and also what time zone people are in and when they are generally available to work.
\end{steps}

If you are attending the lab session live and do not yet have a group, one way the TA could assist is to arrange "speed networking" among those who still need a group. This would involve the TA organizing Zoom Breakout Rooms, where each room is 2-3 students, and each group talks about how they work and what they are looking for in a group member. Then after 5 minutes or so, the Rooms are changed so people are with different people. This could help people get to know each other enough to form lab groups.

\begin{steps}
	\item Once you have a group, meet with each other and decide a) what tools you will use to communicate and collaborate, b) when you will meet, c) what you will do when you need to change an agreement, and d) what you will do when you a person has an issue with how the group is functioning. \textbf{Write this in your lab report. This part counts as data collection and analysis, so it can be identical in each member's report.}
\end{steps}

\section{Zoom}

Zoom is a tool for video conferencing. You probably already it installed.

\begin{steps}
	\item Download and install the application from your app store, or from here: \url{https://zoom.us/download\#client_4meeting}.
	
	\item Some teachers might require students to log in with their uchicago.edu email to access their Zoom meetings, so ensure that you can log in with that email address. You can also change your settings by logging in through the browser: \url{uchicago.zoom.us}
\end{steps}

\section{Spreadsheet software}

Unless you already have data analysis experience with other software, spreadsheet software will be useful for you to collect and analyze data, including plotting and curve fitting. As a UChicago student, you have free access to Microsoft Office 365 Excel (through \url{portal.office.com}). This can work, but is sometimes less intuitive for doing curve fitting than a free open source office software, LibreOffice Calc.

\begin{steps}
	\item Ensure you have access to Microsoft Excel through \url{portal.office.com}
	
	\item (optional, but recommended) Install LibreOffice. You can download it for free from \url{https://www.libreoffice.org/download/download/} (get version 6.3.5, and if you're not sure whether you need the 32-bit or 64-bit version, you almost certainly want the 64-bit version). You can also get it from the Microsoft Store or Mac Store, for a small fee.
\end{steps}

\section{SAOImage DS9}

SAOImage DS9, or DS9 for short, is an image viewer, analyzer, and processor written and used by astronomers for working with astronomical images.

\begin{steps}
	\item Download and install DS9 from \url{http://ds9.si.edu/site/Download.html}.
	\begin{itemize}
		\item For MacOS, unless you know otherwise, choose from the top set of choices (to the right of the blue apple logo). To find your version, from the Apple menu in the corner of the screen, choose ``About This Mac''.
	\end{itemize}
\end{steps}

\section{Flash}

Adobe Flash is software that is currently being phased out, but is still used to enable interactive animations in web browsers.

\begin{steps}
	\item Test if you already have Flash installed by going to \url{http://www.zombo.com/}. If you see some faintly flashing circles and hear an inspirational message welcoming you to zombocom, then Flash is installed and working.
	
	If it does not automatically run, then depending on your browser, try the following:
	\begin{itemize}
		\item Safari: \url{https://helpx.adobe.com/flash-player/kb/enabling-flash-player-safari.html}
		
		\item Microsoft Edge: \url{https://helpx.adobe.com/flash-player/kb/flash-player-issues-windows-10-edge.html}
	\end{itemize}
	
\end{steps}

\section{Report checklist}

Include the following in your lab report. See Appendix~\ref{cha:lab-report-format} for formatting details. Each item below is worth 10 points.

\begin{enumerate}
	\item List of your lab group members and decisions you've made about collaborating (Step 3).
	
	\item A 100--200 word reflection on group dynamics and feedback on the lab manual. Address the following topics: who did what in the lab, how did you work together, what successes and challenges in group functioning did you have, and what would you keep and change about the lab write-up?
\end{enumerate}