\chapter{Peculiarities of light: doppler shift, spectral lines, and limits of resolution}

\section{Looking at Things Far Away}
On the surface, astronomy might seem like a relatively straightforward science. You simply point your telescope at an object and take the measurements you need. However, as is often the case with all areas of study, the reality is often far more complex than it seems. Observations which are simple to make on terrestrial objects suddenly become incredibly difficult to make for an object in space. For instance, how do you measure the velocity of something in space? On the Earth you simply measure how much distance it travels relative to the ground in a set time. When measuring the velocity for astronomical objects, however, the problem becomes a bit more complicated. Sometimes we don't have proper reference frames for the object's motion, other times the object is moving directly away from or towards us so we cant properly tell how much distance its traveled, and so on. Luckily, a lot more information is carried by the light which reaches us than simply how the object looks. In fact, using only very basic concepts of how light is transmitted, we can get very precise values for the velocity of objects in space. By using "light" beyond the visible spectrum detected through radio telescopes, we are able to gather a wealth of information we would not have access to otherwise 


\section{Learning Goals}
\begin{itemize}
	\item Understand how the wavelength and frequency of a wave can be used to calculate the velocity of the source
	
	\item Use physical phenomena to make measurements of quantities which cannot be directly observed
	
	\item Relate the principles of optical telescope to radio telescopes
\end{itemize}

\section{Observation Experiment: Doppler Shift}

\subsection{Goal}
Understand the physics behind the Doppler Effect and be able to come up with a qualitative description of the relationship between the velocity of a source and the perceived frequency of the wave it emits

\subsection{Available Equipment}
\begin{itemize}
	\item Doppler Effect Practical Example: Stationary Car Horn \url{https://www.youtube.com/watch?v=WhjlFSaJhhI&ab_channel=SFXCloud}
	Moving car horn \url{https://www.youtube.com/watch?v=p-hBCcmCUPg&ab_channel=sm1thie}
	\item Tone Generator: \url{https://www.szynalski.com/tone-generator/}
	\item Ripple Tank: \url{https://www.falstad.com/ripple/}
\end{itemize}

\subsection{Steps}

\begin{steps}
	\item Open the two links under ''Doppler Effect Practical Example'' \textbf{These videos feature loud sounds, be sure to change your volume to an acceptable level before opening}. They should lead to youtube video examples of a stationary car horn and a moving car horn.
	
	\item Once you have heard both examples, provide a qualitative description of the differences between both sounds. Write down what you believe might be causing this difference in sound. Write your answers down in the lab report
	
	\item Now, go to the link labeled ''Tone Generator''. This should lead you to a website where you can generate pure sine wave tone. 
	
	\item Using the bar located under the ''play'' button, change the frequency of the tone generated. What relationship do you observe between the frequency of the tone and its pitch? How might this relate to the scenario of the car horns you saw in the example videos. What might be happening to the sound wave of the horn as the car moves. 
	
	\item Open the link labeled ''Ripple Tank'', go to the drop-down menu in the upper-right corner of the window which says ''Example: Single Source'' and change it to ''Doppler Effect 1''. To change the frequency of the source simply move the slider on the right-hand side of the window. To change the velocity of the source, right click on one of the end-nodes (one of the small clear squares in the tank), select edit, and change the ''move  duration'' value. The higher the value, the slower the source, and the lower, the faster the source moves. 
	
	\item By changing the frequency and velocity of the source, see if you can find a relationship between the velocity of the source relative to the sensor, and the frequency of the detected wave. With your group, try and come up with a general mathematical expression which relates the detected frequency $f_{d}$ and the velocity of the source $v_{s}$

\end{steps}

\section{Application Experiment: Redshift and Hydrogen Clouds}

\subsection{Goal}
Using what you found in the Doppler Shift experiment, see if you can devise a method to measure the velocities of astronomical objects

\subsection{Available Equipment}

\begin{itemize}
	\item Openstax Astronomy: Section 5.1 \url{https://openstax.org/books/astronomy/pages/5-1-the-behavior-of-light}
	Section 20.2 \url{https://openstax.org/books/astronomy/pages/20-2-interstellar-gas}
\end{itemize}

\subsection{Steps}
\begin{steps}
	\item Open the first link in the equipment section. It should take you to the section 5.1 of the Openstax Astronomy textbook. Read the subsection titled ''The Wave-Like Characteristics of light''. 
	
	\textbf{The following is adapted from the Center for Astronomy and Physics Education Research's ''Active Learning Tutorials for Astronomy & Planetary Sciences}

	\item %Should I simply provide screenshots of the page or rewrite the activity
	
	\item Now, open the second link which takes you to section 20.2 of the Openstax Astronomy book. Read the subsection ''Neutral Hydrogen Clouds'' %sShould I specify where they should stop reading?
	
	\item Given what you have read and learned so far try in your group to answer the following questions:
	\begin{itemize}
		\item You spot a star which you know should be emitting a signal at 800nm. However, you instead detect a signal at 900nm. What does this tell you about the stars motion? 
		
		\item You spot two gas clouds which should both be emitting at frequencies of around 1400hz. However, for cloud A you detect a signal of 1500hz and for cloud B you detect 1350hz. Which cloud is moving towards you? Away from you? Which one is moving faster
	\end{itemize}
	
	\item In section 20.2 of  Opesntax Astronomy, you should have read and learned about hydrogen clouds and the 21cm line. In your groups, see if you can develop an experimental method for finding the velocity of these hydrogen clouds. Your procedure does not need to incorporate quantitative elements for now.

\end{steps}
	


