
\chapter{Rubrics}\label{cha:rubrics}

%Each lab is graded 50\% on attendance and participation during the lab, and providing evidence in the lab report of completing all steps of the lab, including answering every question. The other 50\% is based on a selection of scientific abilities.

Each scientific ability rubric row assessed is worth a possible 1 point, with ``Missing'' being 0 points, ``Inadequate'' 1/3 points, ``Needs Improvement'' 2/3 points, and ``Adequate'' 1 point.

The scientific abilities rubrics are found on the following pages.

\begin{landscape}

\begin{longtable}{>{\bfseries}p{0.04\textheight}|>{\bfseries\RaggedRight}p{0.23\textheight}|>{\RaggedRight}p{0.21\textheight}|>{\RaggedRight}p{0.21\textheight}|>{\RaggedRight}p{0.22\textheight}|>{\RaggedRight}p{0.22\textheight}}
	\toprule
	& Scientific Ability
	& Missing & Inadequate & Needs Improvement & Adequate \\ \midrule \endhead
	A11
	& Graph
	& No graph is present.
	& A graph is present but the axes are not labeled. There is no scale on the axes.
	& The graph is present and axes are correctly labeled, but the axes do not correspond to the independent and dependent variables, or the scale is not accurate.
	& The graph has correctly labeled axes, independent variable is along the horizontal axis and the scale is accurate.
	\\
	\bottomrule
	\caption{Rubric A: Ability to represent information in multiple ways}\label{rubric:a}
	\end{longtable}

\begin{longtable}{>{\bfseries}p{0.02\textheight}|>{\bfseries\RaggedRight}p{0.25\textheight}|>{\RaggedRight}p{0.21\textheight}|>{\RaggedRight}p{0.21\textheight}|>{\RaggedRight}p{0.22\textheight}|>{\RaggedRight}p{0.22\textheight}}
		\toprule
		& Scientific Ability
		& Missing & Inadequate & Needs Improvement & Adequate \\ \midrule \endhead
		B1
		& Is able to identify the phenomenon to be investigated
		& No phenomenon is mentioned
		& The description of the phenomenon to be investigated is confusing, or it is not the phenomenon of interest.
		& \midsloppy The description of the phenomenon is vague or incomplete.
		& The phenomenon to be investigated is clearly stated. \\ \midrule
		B2
		& Is able to design a reliable experiment that investigates the phenomenon
		& The experiment does not investigate the phenomenon.
		& The experiment may not yield any interesting patterns.
		& Some important aspects of the phenomenon will not be observable.
		& The experiment might yield interesting patterns relevant to the investigation of the phenomenon. \\ \midrule
		B3
		& Is able to decide what physical quantities are to be measured and identify independent and dependent variables
		& The physical quantities are irrelevant.
		& Only some of physical quantities are relevant.
		& The physical quantities are relevant. However, independent and dependent variables are not identified.
		& The physical quantities are relevant and independent and dependent variables are identified. \\ \midrule
		B4
		& Is able to describe how to use available equipment to make measurements
		& At least one of the chosen measurements cannot be made with the available equipment.
		& All chosen measurements can be made, but no details are given about how it is done.
		& All chosen measurements can be made, but the details of how it is done are vague or incomplete.
		& All chosen measurements can be made and all details of how it is done are clearly provided. \\ \midrule
		B5
		& Is able to describe what is observed without trying to explain, both in words and by means of a picture of the experimental setup.
		& No description is mentioned.
		& A description is incomplete. No labeled sketch is present. Or, observations are adjusted to fit expectations.
		& A description is complete, but mixed up with explanations or pattern. Or the sketch is present but is difficult to understand.
		& Clearly describes what happens in the experiments both verbally and with a sketch. Provides other representations when necessary (tables and graphs). \\ \midrule
		B6
		& Is able to identify the shortcomings in an experiment and suggest improvements
		& No attempt is made to identify any shortcomings of the experiment.
		& The shortcomings are described vaguely and no suggestions for improvement are made.
		& Not all aspects of the design are considered in terms of shortcomings or improvements.
		& All major shortcomings of the experiment are identified and reasonable suggestions for improvement are made. \\ \midrule
		B7
		& Is able to identify a pattern in the data
		& No attempt is made to search for a pattern.
		& The pattern described is irrelevant or inconsistent with the data.
		& The pattern has minor errors or omissions. Terms like ``proportional'' used without clarity, e.g.\ is the proportionality linear, quadratic, etc.
		& The pattern represents the relevant trend in the data. When possible, the trend is described in words. \\ \midrule
		B8
		& Is able to represent a pattern mathematically (if applicable)
		& No attempt is made to represent a pattern mathematically.
		& The mathematical expression does not represent the trend.
		& No analysis of how well the expression agrees with the data is included, or some features of the pattern are missing.
		& The expression represents the trend completely and an analysis of how well it agrees with the data is included. \\ \midrule
		B9
		& Is able to devise an explanation for an observed pattern
		& No attempt is made to explain the observed pattern.
		& An explanation is vague, not testable, or contradicts the pattern.
		& An explanation contradicts previous knowledge or the reasoning is flawed.
		& A reasonable explanation is made. It is testable and it explains the observed pattern. \\
		\bottomrule
		\caption{Rubric B: Ability to design and conduct an observational experiment \cite{etkina_scientific_2006}.}\label{rubric:b}
	\end{longtable}

\begin{longtable}{>{\bfseries}p{0.02\textheight}|>{\bfseries\RaggedRight}p{0.25\textheight}|>{\RaggedRight}p{0.21\textheight}|>{\RaggedRight}p{0.21\textheight}|>{\RaggedRight}p{0.22\textheight}|>{\RaggedRight}p{0.22\textheight}}
	\toprule
	& Scientific Ability
	& Missing & Inadequate & Needs Improvement & Adequate \\ \midrule \endhead
	C1
	& Is able to identify the hypothesis to be tested
	& No mention is made of a hypothesis.
	& An attempt is made to identify the hypothesis to be tested but it is described in a confusing manner.
	& The hypothesis to be tested is described but there are minor omissions or vague details.
	& The hypothesis is clearly, specifically, and thoroughly stated.
	\\ \midrule
	C2
	& Is able to design a reliable experiment that tests the hypothesis
	& The experiment does not test the hypothesis.
	& The experiment tests the hypothesis, but due to the nature of the design it is likely the data will lead to an incorrect judgment.
	& The experiment tests the hypothesis, but due to the nature of the design there is a moderate chance the data will lead to an inconclusive judgment.
	& The experiment tests the hypothesis and has a high likelihood of producing data that will lead to a conclusive judgment.
	\\ \midrule
	C4
	& Is able to make a reasonable prediction based on a hypothesis
	& No prediction is made. The experiment is not treated as a testing experiment.
	& A prediction is made, but it is identical to the hypothesis, OR prediction is made based on a source unrelated to the hypothesis being tested, or is completely inconsistent with hypothesis being tested, OR prediction is unrelated to the context of the designed experiment.
	& Prediction follows from hypothesis but is flawed because relevant assumptions are not considered, OR prediction is incomplete or somewhat inconsistent with hypothesis, OR prediction is somewhat inconsistent with the experiment.
	& A prediction is made that follows from hypothesis, is distinct from the hypothesis, accurately describes the expected outcome of the experiment, and incorporates relevant assumptions if needed.
	\\ \midrule
	C5
	& Is able to identify the assumptions made in making the prediction
	& No attempt is made to identify assumptions.
	& An attempt is made to identify assumptions, but the assumptions are irrelevant or are confused with the hypothesis.
	& Relevant assumptions are identified but are not significant for making the prediction.
	& Sufficient assumptions are correctly identified, and are significant for the prediction that is made.
	\\ \midrule
	C6
	& Is able to determine specifically the way in which assumptions might affect the prediction
	& No attempt is made to determine the effects of assumptions.
	& The effects of assumptions are mentioned but are described vaguely.
	& The effects of assumptions are determined, but no attempt is made to validate them.
	& The effects of assumptions are determined and the assumptions are validated.
	\\ \midrule
	C7
	& Is able to decide whether the prediction and the outcome agree/disagree
	& No mention of whether the prediction and outcome agree/disagree.
	& A decision about the agreement/disagreement is made but is not consistent with the results of the experiment.
	& A reasonable decision about the agreement/disagreement is made but experimental uncertainty is not taken into account.
	& A reasonable decision about the agreement/disagreement is made and experimental uncertainty is taken into account.
	\\ \midrule
	C8
	& Is able to make a reasonable judgment about the hypothesis
	& No judgment is made about the hypothesis.
	& A judgment is made but is not consistent with the outcome of the experiment.
	& A judgment is made, is consistent with the outcome of the experiment, but assumptions are not taken into account.
	& A judgment is made, is consistent with the outcome of the experiment, and assumptions are taken into account.
	\\ \bottomrule
	\caption{Rubric C: Ability to design and conduct a testing experiment \cite{etkina_scientific_2006}.}\label{rubric:c}
\end{longtable}

\begin{longtable}{>{\bfseries}p{0.02\textheight}|>{\bfseries\RaggedRight}p{0.25\textheight}|>{\RaggedRight}p{0.21\textheight}|>{\RaggedRight}p{0.21\textheight}|>{\RaggedRight}p{0.22\textheight}|>{\RaggedRight}p{0.22\textheight}}
	\toprule
	& Scientific Ability
	& Missing & Inadequate & Needs Improvement & Adequate \\ \midrule \endhead
	D1
	& Is able to identify the problem to be solved
	& No mention is made of the problem to be solved.
	& An attempt is made to identify the problem to be solved but it is described in a confusing manner.
	& The problem to be solved is described but there are minor omissions or vague details.
	& The problem to be solved is clearly stated.
	\\ \midrule
	D2
	& Is able to design a reliable experiment that solves the problem.
	& The experiment does not solve the problem.
	& The experiment attempts to solve the problem but due to the nature of the design the data will not lead to a reliable solution.
	& The experiment attempts to solve the problem but due to the nature of the design there is a moderate chance the data will not lead to a reliable solution.
	& The experiment solves the problem and has a high likelihood of producing data that will lead to a reliable solution.
	\\ \midrule
	D3
	& Is able to use available equipment to make measurements
	& At least one of the chosen measurements cannot be made with the available equipment.
	& All of the chosen measurements can be made, but no details are given about how it is done.
	& All of the chosen measurements can be made, but the details about how they are done are vague or incomplete.
	& All of the chosen measurements can be made and all details about how they are done are provided and clear.
	\\ \midrule
	D4
	& Is able to make a judgment about the results of the experiment
	& No discussion is presented about the results of the experiment.
	& A judgment is made about the results, but it is not reasonable or coherent.
	& An acceptable judgment is made about the result, but the reasoning is incomplete, OR uncertainties are not taken into account, OR assumptions are not discussed, OR the result is written as a single number.
	& An acceptable judgment is made about the result, with clear reasoning. The effects of assumptions and experimental uncertainties are considered. The result is written as an interval.
	\\ \midrule
	D5
	& Is able to evaluate the results by means of an independent method
	& No attempt is made to evaluate the consistency of the result using an independent method.
	& A second independent method is used to evaluate the results. However there is little or no discussion about the differences in the results due to the two methods.
	& A second independent method is used to evaluate the results. The results of the two methods are compared correctly using experimental uncertainties. But there is little or no discussion of the possible reasons for the differences when the results are different.
	& A second independent method is used to evaluate the results and the evaluation is correctly done with the experimental uncertainties. The discrepancy between the results of the two methods, and possible reasons are discussed.
	\\ \midrule
	D7
	& Is able to choose a productive mathematical procedure for solving the experimental problem
	& Mathematical procedure is either missing, or the equations written down are irrelevant to the design.
	& A mathematical procedure is described, but is incorrect or incomplete, due to which the final answer cannot be calculated. Or units are inconsistent.
	& Correct and complete mathematical procedure is described but an error is made in the calculations. All units are consistent.
	& Mathematical procedure is fully consistent with the design. All quantities are calculated correctly with proper units. Final answer is meaningful.
	\\ \midrule
	D8
	& Is able to identify the assumptions made in using the mathematical procedure
	& No attempt is made to identify any assumptions.
	& An attempt is made to identify assumptions, but the assumptions are irrelevent or incorrect for the situation.
	& Relevant assumptions are identified but are not significant for solving the problem.
	& All relevant assumptions are correctly identified.
	\\ \bottomrule
	\caption{Rubric D: Ability to design and conduct an application experiment \cite{etkina_scientific_2006}.}\label{rubric:d}
\end{longtable}

\begin{longtable}{>{\bfseries}p{0.02\textheight}|>{\bfseries\RaggedRight}p{0.25\textheight}|>{\RaggedRight}p{0.21\textheight}|>{\RaggedRight}p{0.21\textheight}|>{\RaggedRight}p{0.22\textheight}|>{\RaggedRight}p{0.22\textheight}}
	\toprule
	& Scientific Ability
	& Missing & Inadequate & Needs Improvement & Adequate \\ \midrule \endhead	
	F1
	& Is able to communicate the details of an experimental procedure clearly and completely
	& Diagrams are missing and/or experimental procedure is missing or extremely vague.
	& Diagrams are present but unclear and/or experimental procedure is present but important details are missing. It takes a lot of effort to comprehend.
	& Diagrams and/or experimental procedure are present and clearly labeled but with minor omissions or vague details. The procedure takes some effort to comprehend.
	& Diagrams and/or experimental procedure are clear and complete. It takes no effort to comprehend.
	\\ \midrule
	F2
	& Is able to communicate the point of the experiment clearly and completely
	& No discussion of the point of the experiment is present.
	& The experiment and findings are discussed but vaguely. There is no reflection on the quality and importance of the findings.
	& The experiment and findings are communicated but the reflection on their importance and quality is not present.
	& The experiment and findings are discussed clearly. There is deep reflection on the quality and importance of the findings.
	\\
	\bottomrule
	\caption{Rubric F: Ability to communicate scientific ideas \cite{etkina_scientific_2006}.}\label{rubric:f}
\end{longtable}

\begin{longtable}{>{\bfseries}p{0.02\textheight}|>{\bfseries\RaggedRight}p{0.25\textheight}|>{\RaggedRight}p{0.21\textheight}|>{\RaggedRight}p{0.21\textheight}|>{\RaggedRight}p{0.22\textheight}|>{\RaggedRight}p{0.22\textheight}}
	\toprule
	& Scientific Ability
	& Missing & Inadequate & Needs Improvement & Adequate \\ \midrule \endhead	
	G1
	& Is able to identify sources of experimental uncertainty
	& No attempt is made to identify experimental uncertainties.
	& An attempt is made to identify experimental uncertainties, but most are missing, described vaguely, or incorrect.
	& Most experimental uncertainties are correctly identified. But there is no distinction between random and instrumental uncertainty.
	& All experimental uncertainties are correctly identified. There is a distinction between instrumental and random uncertainty.
	\\ \midrule
	G2
	& Is able to evaluate specifically how identified experimental uncertainties affect the data
	& No attempt is made to evaluate experimental uncertainties.
	& An attempt is made to evaluate uncertainties, but most are missing, described vaguely, or incorrect. Or the final result does not take uncertainty into account.
	& The final result does take the identified uncertainties into account but is not correctly evaluated. Uncertainty propagation is not used or is used incorrectly.
	& The experimental uncertainty of the final result is correctly evaluated. Uncertainty propagation is used appropriately.
	\\ \midrule
	G3
	& Is able to describe how to minimize experimental uncertainty and actually do it
	& No attempt is made to describe how to minimize experimental uncertainty and no attempt to minimize is present.
	& A description of how to minimize experimental uncertainty is present, but there is no attempt to actually minimize it.
	& An attempt is made to minimize the uncertainty in the final result is made but the method is not very effective.
	& The uncertainty is minimized in an effective way.
	\\ \midrule
	G4
	& Is able to record and represent data in a meaningful way
	& Data are either absent or incomprehensible.
	& Some important data are absent or incomprehensible. They are not organized in tables or the tables are not labeled properly.
	& All important data are present, but recorded in a way that requires some effort to comprehend. The tables are labeled but labels are confusing.
	& All important data are present, organized, and recorded clearly. The tables are labeled and placed in a logical order.
	\\ \midrule
	G5
	& Is able to analyze data appropriately
	& No attempt is made to analyze the data.
	& An attempt is made to analyze the data, but it is either seriously flawed or inappropriate.
	& The analysis is appropriate but it contains errors or omissions.
	& The analysis is appropriate, complete, and correct.
	\\
	\bottomrule
	\caption{Rubric G: Ability to collect and analyze experimental data \cite{etkina_scientific_2006}.}\label{rubric:g}
\end{longtable}

\end{landscape}